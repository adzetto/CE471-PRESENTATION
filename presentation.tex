\documentclass{beamer}
\usetheme{Madrid}
\usecolortheme{default}

\usepackage[utf8]{inputenc}
\usepackage{graphicx}
\usepackage{amsmath}

% Presentation Info
\title[Hydrokinetic Energy]{Hydrokinetic Energy Conversion Systems}
\subtitle{Group 10 Project}
\author[Group 10]{Hamidreza Khalaj Zahrari \and Muhammet Yağcıoğlu}
\institute[]{Civil Engineering Department}
\date{05.01.2026}

\begin{document}

% Slide 1: Title
\begin{frame}
    \titlepage
\end{frame}

% Slide 2: Introduction
\begin{frame}{What is Hydrokinetic Energy?}
    \begin{itemize}
        \item \textbf{Definition:} Energy derived from the motion of water (kinetic energy) rather than its potential energy (head).
        \item \textbf{Analogy:} "Underwater Wind Turbines" - utilizing water density instead of air.
        \item \textbf{Key Difference:}
        \begin{itemize}
            \item \textbf{Conventional Hydro:} Uses potential energy (Dam height).
            \item \textbf{Hydrokinetic:} Uses kinetic energy (Flow velocity).
        \end{itemize}
    \end{itemize}
\end{frame}

% Slide 3: History
\begin{frame}{Historical Context}
    \begin{itemize}
        \item \textbf{Ancient Origins:} Water wheels (M.Ö. 4000) used for grinding grain were the first use of kinetic water energy.
        \item \textbf{Modern Evolution (1920s):} Georges Darrieus proposed vertical axis turbines for wind, which was later adapted for water.
        \item \textbf{Current Era:} Combining ancient principles with modern aerodynamics and hydrodynamics (submarine tech).
    \end{itemize}
\end{frame}

% Slide 4: Working Principle
\begin{frame}{Working Principle}
    Power generation is based on the kinetic energy equation:
    \begin{equation}
        P = \frac{1}{2} \rho A v^3 C_p
    \end{equation}
    \begin{itemize}
        \item $P$: Power output (Watts)
        \item $\rho$: Water density ($\approx 1000 \text{ kg/m}^3$, 830x denser than air)
        \item $A$: Swept area ($m^2$)
        \item $v$: Flow velocity ($m/s$) - \textbf{Crucial Factor ($v^3$)}
        \item $C_p$: Power coefficient
    \end{itemize}
\end{frame}

% Slide 5: Comparison
\begin{frame}{Comparison with Conventional Dams}
    \begin{table}
        \begin{tabular}{|l|l|l|}
            \hline
            \textbf{Feature} & \textbf{Conventional Dams} & \textbf{Hydrokinetic} \\
            \hline
            \textbf{Source} & Potential Energy (Head) & Kinetic Energy (Velocity) \\
            \hline
            \textbf{Infrastructure} & Massive Concrete Dams & Modular Turbines \\
            \hline
            \textbf{Env. Impact} & Flooding, Fish Barriers & Minimal, Fish Friendly \\
            \hline
            \textbf{Installation} & Years (5-10 years) & Weeks (Modular) \\
            \hline
        \end{tabular}
    \end{table}
\end{frame}

% Slide 6: Turkey Context
\begin{frame}{Hydrokinetic Energy in Turkey}
    Turkey has significant potential in straits and irrigation channels.
    \begin{block}{Key Projects \& Initiatives}
        \begin{itemize}
            \item \textbf{Mavi İda Enerji (Çanakkale):} Developing the "M30" domestic current turbine for the Bosphorus/Dardanelles.
            \item \textbf{Adana Irrigation Channels:} Pilot projects using portable turbines in DSİ channels (ECC Makina).
            \item \textbf{TOBB ETÜ (Ankara):} Developing "Container Type HPP" and in-pipe turbines for municipal water lines (İZSU).
        \end{itemize}
    \end{block}
\end{frame}

% Slide 7: Potential & Challenges
\begin{frame}{Potential \& Challenges in Turkey}
    \begin{columns}
        \begin{column}{0.5\textwidth}
            \textbf{Potential}
            \begin{itemize}
                \item \textbf{The Straits (Bosphorus):} A natural energy corridor with continuous two-way currents.
                \item \textbf{Rivers:} Euphrates (Fırat) and Tigris (Dicle).
                \item \textbf{Irrigation:} GAP region channels.
            \end{itemize}
        \end{column}
        \begin{column}{0.5\textwidth}
            \textbf{Challenges}
            \begin{itemize}
                \item \textbf{Maritime Traffic:} Busy shipping lanes in Bosphorus.
                \item \textbf{Anchoring:} Difficult seabed conditions.
                \item \textbf{Maintenance:} Biofouling and underwater access.
            \end{itemize}
        \end{column}
    \end{columns}
\end{frame}

% Slide 8: Civil Engineering
\begin{frame}{Civil Engineering Aspects}
    \begin{itemize}
        \item \textbf{Site Selection:} Bathymetry mapping and ADCP velocity surveys.
        \item \textbf{Foundation Design:}
        \begin{itemize}
            \item Gravity Base (Concrete weight)
            \item Monopiles (Driven steel piles)
            \item Mooring Systems (Anchors \& Cables)
        \end{itemize}
        \item \textbf{Grid Connection:} Subsea cabling and onshore substations.
    \end{itemize}
\end{frame}

% Slide 9: Conclusion
\begin{frame}{Summary}
    \begin{itemize}
        \item Hydrokinetic energy is a sustainable alternative to dams, utilizing water velocity ($P \propto v^3$).
        \item Offers high energy density and predictability.
        \item \textbf{Turkey's Role:} Significant potential in Bosphorus and irrigation channels with ongoing R\&D (Mavi İda, TOBB ETÜ).
        \item Civil engineers are vital for infrastructure, foundation, and deployment.
    \end{itemize}
    \vspace{1cm}
    \centering
    \textbf{\Large Thank You}
\end{frame}

\end{document}
