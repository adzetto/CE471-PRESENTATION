\documentclass{beamer}
\usetheme{Madrid}
\usecolortheme{beaver}
\usefonttheme{serif} % Use Computer Modern Serif (LaTeX original look)
\usepackage{lmodern}

\usepackage[utf8]{inputenc}
\usepackage{graphicx}

% Global centering for frame content
\setbeamertemplate{frametitle}[default][center]
\makeatletter
\newcommand*{\centerall}{\centering}
\makeatother

% Presentation Info
\title[Hydrokinetic Energy]{Hydrokinetic Energy Conversion Systems}
\subtitle{A Historical and Functional Overview}
\author[Group 10]{Hamidreza Khalaj Zahrari \and Muhammet Yağcıoğlu}
\institute[IZTECH]{
    \textbf{Izmir Institute of Technology} \\
    Civil Engineering Department
}
\date{January 5, 2026}

\begin{document}

% Slide 1
\begin{frame}
    \titlepage
\end{frame}

% Slide 2
\begin{frame}{Introduction}
    \centering
    Hydrokinetic energy conversion is a method of generating electricity from the movement of water without the need for large dams or reservoirs.
    
    \vspace{1em}
    
    Unlike traditional hydropower, which relies on the potential energy of falling water, hydrokinetic systems utilize the kinetic energy of flowing water in rivers, estuaries, and ocean currents.
\end{frame}

% Slide 3
\begin{frame}{Why Do We Need Hydrokinetic Energy?}
    \centering
    The global demand for energy is rising exponentially.
    
    \vspace{1em}
    
    Fossil fuels are finite and contribute significantly to global warming and climate change.
    
    \vspace{1em}
    
    We need sustainable, renewable energy sources that can provide consistent power generation to support modern civilization.
\end{frame}

% Slide 4
\begin{frame}{The Limitations of Solar and Wind}
    \centering
    While solar and wind energy are popular, they are intermittent.
    
    \vspace{1em}
    
    The sun does not shine at night, and the wind does not always blow.
    
    \vspace{1em}
    
    This intermittency requires expensive battery storage solutions or backup power plants to ensure grid stability.
\end{frame}

% Slide 5
\begin{frame}{The Advantage of Water Currents}
    \centering
    Water currents are remarkably predictable and consistent.
    
    \vspace{1em}
    
    Tides can be predicted years in advance, and river flows are generally stable or change slowly.
    
    \vspace{1em}
    
    This predictability makes hydrokinetic energy an excellent candidate for baseload power generation, reducing the need for energy storage.
\end{frame}

% Slide 6
\begin{frame}{History of Water Energy: Ancient Times}
    \centering
    The use of water energy dates back thousands of years.
    
    \vspace{1em}
    
    Ancient civilizations, including the Greeks and Romans, used water wheels to grind grain and pump water.
    
    \vspace{1em}
    
    These early systems were the ancestors of modern hydrokinetic turbines, utilizing the flow of rivers directly.
\end{frame}

% Slide 7
\begin{frame}{History: The Industrial Revolution}
    \centering
    During the Industrial Revolution, water power was the primary source of energy for factories and textile mills.
    
    \vspace{1em}
    
    Water wheels evolved into more efficient turbines.
    
    \vspace{1em}
    
    However, as electricity demand grew, the focus shifted to massive hydroelectric dams that could store vast amounts of water.
\end{frame}

% Slide 8
\begin{frame}{The Problem with Large Dams}
    \centering
    While effective, large dams have significant environmental downsides.
    
    \vspace{1em}
    
    They disrupt local ecosystems, block fish migration, and alter sediment transport in rivers.
    
    \vspace{1em}
    
    Constructing dams also requires displacing communities and flooding large areas of land.
\end{frame}

% Slide 9
\begin{frame}{Return to Hydrokinetic Concepts}
    \centering
    In recent decades, there has been a renewed interest in "run-of-river" and zero-head systems.
    
    \vspace{1em}
    
    Engineers realized that we could generate power without building dams by placing turbines directly into the flow.
    
    \vspace{1em}
    
    This approach minimizes environmental impact and civil infrastructure costs.
\end{frame}

% Slide 10
\begin{frame}{How It Works: The Concept}
    \centering
    The operating principle is very similar to a wind turbine.
    
    \vspace{1em}
    
    Instead of air moving over blades to create lift and rotation, moving water flows over underwater blades.
    
    \vspace{1em}
    
    Since water is much denser than air, even slow-moving water contains a tremendous amount of energy.
\end{frame}

% Slide 11
\begin{frame}{System Components}
    \centering
    A typical system consists of three main parts:
    
    \vspace{1em}
    
    \textbf{The Rotor:} Blades that capture the energy from the water.
    
    \vspace{1em}
    
    \textbf{The Generator:} Converts the mechanical rotation into electricity.
    
    \vspace{1em}
    
    \textbf{The Mooring/Foundation:} Holds the turbine in place against the strong current.
\end{frame}

% Slide 12
\begin{frame}{Types of Turbines: Horizontal Axis}
    \centering
    Horizontal Axis Hydrokinetic Turbines (HAHT) look like underwater windmills.
    
    \vspace{1em}
    
    They are the most common design and are highly efficient.
    
    \vspace{1em}
    
    However, they require the mechanism to be oriented into the direction of the flow.
\end{frame}

% Slide 13
\begin{frame}{Types of Turbines: Vertical Axis}
    \centering
    Vertical Axis Hydrokinetic Turbines (VAHT) look like eggbeaters.
    
    \vspace{1em}
    
    Their main advantage is that they can capture flow from any direction without needing to reorient.
    
    \vspace{1em}
    
    This makes them ideal for areas where tide directions change frequently.
\end{frame}

% Slide 14
\begin{frame}{Civil Engineering Aspects: Foundations}
    \centering
    Securing these turbines is a major civil engineering challenge.
    
    \vspace{1em}
    
    Gravity-based foundations use heavy concrete blocks to hold the turbine down by weight.
    
    \vspace{1em}
    
    Pile foundations are driven deep into the riverbed or seabed, similar to offshore oil platforms.
\end{frame}

% Slide 15
\begin{frame}{Civil Engineering Aspects: Materials}
    \centering
    The underwater environment is harsh.
    
    \vspace{1em}
    
    Materials must withstand constant moisture, high pressure, and the corrosive effects of saltwater (in marine applications).
    
    \vspace{1em}
    
    Special marine-grade concrete and composites are essential for longevity.
\end{frame}

% Slide 16
\begin{frame}{Environmental Impact}
    \centering
    Hydrokinetic systems are considered environmentally friendly.
    
    \vspace{1em}
    
    The blades rotate slowly, allowing fish to swim around them safely.
    
    \vspace{1em}
    
    There is no thermal pollution, and no emissions are released into the water or air.
\end{frame}

% Slide 17
\begin{frame}{Potential in Turkey}
    \centering
    Turkey has significant potential for hydrokinetic energy.
    
    \vspace{1em}
    
    The Bosphorus and Dardanelles straits have strong, continuous currents that are ideal for energy harvesting.
    
    \vspace{1em}
    
    Additionally, the extensive irrigation canal networks in the GAP region offer controlled environments for smaller turbines.
\end{frame}

% Slide 18
\begin{frame}{Economic Viability}
    \centering
    Currently, the cost of energy is higher than established technologies like wind or solar.
    
    \vspace{1em}
    
    However, as the technology matures and mass production begins, costs are expected to drop significantly.
    
    \vspace{1em}
    
    The low maintenance and long lifespan of civil structures contribute to good long-term economics.
\end{frame}

% Slide 19
\begin{frame}{Future Outlook}
    \centering
    The future of hydrokinetic energy is promising.
    
    \vspace{1em}
    
    Research is focusing on array configurations, where multiple turbines work together like a wind farm.
    
    \vspace{1em}
    
    Innovations in materials science and smart grid integration will further enhance their adoption.
\end{frame}

% Slide 20
\begin{frame}{Conclusion}
    \centering
    Hydrokinetic energy conversion is a vital piece of the sustainable energy puzzle.
    
    \vspace{1em}
    
    It offers a reliable, baseload power source that complements intermittent renewables.
    
    \vspace{1em}
    
    For civil engineers, it presents exciting challenges in structural design, fluid dynamics, and marine construction.
    
    \vspace{2em}
    \textbf{Thank You}
\end{frame}

\end{document}
